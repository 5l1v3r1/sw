% !TEX root = sw.tex
% !TeX spellcheck = en_US

\chapter*{Introduction}

\section*{This document}

This document describes a modern approach to software management.
It consists of several chapters which include the new approach itself, documentation for client-side tools and utilities and server-side (website) functionality description.

License for this documentation is unspecified yet.
% GNU FDL?
% CC?

\section*{Overview}

Software Network is a project dedicated to better software management.

Originally started as a package manager ([CPPAN](https://github.com/cppan/cppan)) to C/C++ languages based on CMake build system, it is evolved to independent set of tools and libraries.
CPPAN or v1 was a playground where different ideas were studied and checked.

Main user tool is called `sw`. Whole project also may be called as SW. Pronounce it as you like: `[software]` or `[sw]` or `[sv]`.

\section*{Useful links}

\begin{enumerate}

\item
* Website - \url{https://software-network.org/}

\item
* Download client - \url{https://software-network.org/client/}

\item
* Command Line Reference \url{https://github.com/SoftwareNetwork/sw/wiki/Command-Line-Reference}

\item
* Publish a project](https://github.com/SoftwareNetwork/sw/wiki/Publish)

\item
* Why SW? Check out its [Features](https://github.com/SoftwareNetwork/sw/wiki/Features)

\item
* F.A.Q.](https://github.com/SoftwareNetwork/sw/wiki/F.A.Q.)

\item
* History](https://github.com/SoftwareNetwork/sw/wiki/History)

\item
* About](https://github.com/SoftwareNetwork/sw/wiki/About)

\end{enumerate}



\chapter{Software Management}

\section{x}

\section{x2}

\chapter{SW client tool}

\section{Quick Start}

\section{Command Line Reference}



\chapter{Software Network -- server frontend}
