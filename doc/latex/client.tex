% !TEX root = sw.tex
% !TeX spellcheck = en_US


\chapter{Client tools}

Main client tool is SW.

A gui frontend is also available. It is Qt-based and widgets-based (no QML).

\section{Quick Start}



\subsection{Package Management}

SW uses packages paths to uniquely identify projects on the web.
Project path is similar to Java packages. Version is appended to it.

`org.username.path.to.project.target-version`

It contains:
\begin{enumerate}
\item
Namespace: `org`
\item
User name: `username`
\item
Path to target: `path.to.project.target`
\end{enumerate}

Allowed namespaces are:
\begin{enumerate}
\item
`org` - organization's public projects with open (free) software license.
\item
`com` - organization's projects with commercial license.
\item
`pub` - user's public projects.
\item
`pvt` - user's private projects.
\end{enumerate}

`version` - is a **semver-like** version.

\subsection{Build process}

SW uses configuration files to describe build process (Build).

Build contains several solutions (Solution).
Each Solution is responsible for exactly one configuration. For example, x86+dll+Release or arm64+static+Debug (plus sign is used only for this document).

Solution consists of targets, projects and directories. Each object might include any other object.

SW uses multipass approach if it needs to build more than one solution.
(Unlike CMake that is telled which options go to what configuration during single pass.)

It is possible to use different frontends to describe build.
Main frontend is called `sw` as well as the program and uses C++ code directly.
Current secondary frontend is `cppan` YAML frontend used in first version of the program.

\subsection{Frontends}

\subsubsection{sw}

_HINT: run `sw create project` to force sw to create first config and generate an IDE project for you._

To start your first build configuration, create a file called `sw.cpp`.

\begin{verbatim}
// this is our 'main' build function signature
// function accepts current Solution
void build(Solution &s)
{
    // add non copyable target to Solution of type Executable with name 'mytarget'
    auto &t = s.addTarget<Executable>("mytarget");
    t += "src/main.cpp"; // add our first source file
    // t += "dependency.path"_dep; // add needed dependency
}
\end{verbatim}

SW uses the latest current standard available - C++17.

To control your build (add definitions, dependencies, link libraries) sw uses user-defined literals. For more help on this, see https://github.com/SoftwareNetwork/sw/blob/master/src/sw/driver/suffix.h
Examples:

\begin{verbatim}
t += "MY_DEFINITION"_def;
t += "ws2_32.lib"_lib;
t += "src"_idir;
\end{verbatim}

Now create `src/main.cpp` file with contents you want.

Run `sw build` to perform build or `sw generate` to generate IDE project.



Visibility

To provide control on properies visibility, one might use following target members:

\begin{verbatim}
t.Private += ...; // equals to t += ...;
t.Protected += ...; // available only for this target and current project members
t.Public += ...; // available for this target and all downstream users
t.Inheritance += ...; // available only for all downstream users
\end{verbatim}

Underlying implementation uses three bits to describe inheritance.

* 1 - This target T
* 2 - This project P (excluding T)
* 3 - All other projects A (excluding P and T)

This gives 7 different inheritance modes (excluding 000) and you could use them all!





More than one configuration

To create additional configs you might add them on command line or write into configuration.
Add following code to your `sw.cpp`:

\begin{verbatim}
// configure signature
// Note that it accepts whole build, not a single solution!
void configure(Build &b)
{
    auto &s1 = b.addSolution(); // add first solution
    s1.Settings.Native.LibrariesType = LibraryType::Static;
    s1.Settings.Native.ConfigurationType = ConfigurationType::ReleaseWithDebugInformation;

    auto &s2 = b.addSolution(); // add first solution
    s2.Settings.Native.LibrariesType = LibraryType::Shared;
    s2.Settings.Native.ConfigurationType = ConfigurationType::Debug;

    return;

    // following code can be used for advanced toolchain configuration
    if (b.isConfigSelected("cygwin2macos"))
        b.loadModule("utils/cc/cygwin2macos.cpp").call<void(Solution&)>("configure", b);
    else if (b.isConfigSelected("win2macos"))
        b.loadModule("utils/cc/win2macos.cpp").call<void(Solution&)>("configure", b);
}
\end{verbatim}

In this function you could provide full custom toolchain to use (compilers, linkers, librarians etc.).
Extended configs can be found here: https://github.com/SoftwareNetwork/sw/tree/master/utils/cc


\subsubsection{cppan}








\section{Command Line Reference}




\section{Supported Languages}

Mature support:
\begin{enumerate}
\item ASM
\item C
\item C++
\end{enumerate}

Hello-world support:
\begin{enumerate}
\item C\#
\item D
\item Fortran
\item Go
\item Java
\item Kotlin
\item Rust
\end{enumerate}


Ported Intepreters

- perl (partial)

- python

- tcl

Work started

- java


\section{Integrations}

It is possible to use SW as package manager for your favourite build system.

Here you can find list of completed integrations with examples.

\subsection{CMake}


[Full example](https://github.com/SoftwareNetwork/sw/blob/master/test/integrations/CMakeLists.txt)

Steps
\begin{enumerate}
\item
Download SW, unpack and add to PATH.
\item
Run `sw setup` to perform initial integration into system.
\item
Add necessary parts to your CMakeLists.txt (see below).
\end{enumerate}

First, make sure CMake is able to found SW package: ```find_package(SW REQUIRED)```

Then add necessary dependencies (packages) with `sw_add_package()`. You can omit version. In this case the latest will be used.

Next step is to execute SW to prepare imported targets script with `sw_execute()`.

\begin{verbatim}
find_package(SW REQUIRED)
sw_add_package(
    org.sw.demo.sqlite3
    org.sw.demo.glennrp.png
)
sw_execute()
\end{verbatim}

Last thing is to add dependency to your package:

\begin{verbatim}
add_executable(mytarget ${MY_SOURCES})
target_link_libraries(mytarget 
    org.sw.demo.glennrp.png
)
\end{verbatim}

Configure SW deps

Use `set(SW_BUILD_SHARED_LIBS 1)` to build shared libs instead of static (default).

\subsection{Waf}

[Full example](https://github.com/SoftwareNetwork/sw/blob/master/test/integrations/wscript)

\begin{enumerate}
\item
Copy-paste these lines to your file \url{https://github.com/SoftwareNetwork/sw/blob/master/test/integrations/wscript#L4-L25}

\item
Replace these lines with your dependencies \url{https://github.com/SoftwareNetwork/sw/blob/master/test/integrations/wscript#L9-L10}

\item
Use deps like on those lines \url{https://github.com/SoftwareNetwork/sw/blob/master/test/integrations/wscript#L35-L36}

\end{enumerate}

To build things call

```
waf sw_configure sw_build configure build
```

`sw_configure` setups SW for waf.

`sw_build` builds necessary deps.





\section{Generators}

SW allows you to generate projects to many different existing systems.

In square brackets current support status is shown.

\begin{enumerate}
\item
* Visual Studio (C++ projects) [2/5]
\item
* Visual Studio (NMake projects) [4/5]
\item
* Visual Studio (Utility projects) [3/5]
\item
* Visual Studio (NMake + Utility projects) [3/5]
\item
* CMake [0/5]
\item
* CompilationDatabase [5/5]
\item
* Make [5/5]
\item
* NMake [4/5]
\item
* Ninja [5/5]
\item
* QMake [0/5]
\item
* Batch [5/5]
\item
* Shell [5/5]
\end{enumerate}


\section{Command Line Reference}

\subsection{sw build}

Build current project.

To get build and dependencies graphs, run:
```
sw -print-graph build
```

\subsection{sw generate}

Generate IDE project.

\subsection{sw upload}

Upload current project to Software Network.
Project will be fetched from your version control system using specified tag/version/branch.

\subsection{sw create}

Create templated project.

\subsection{sw -help}

CLI help.

\begin{verbatim}
OVERVIEW: SW: Software Network Client

  SW is a Universal Package Manager and Build System

USAGE: sw.client.sw-0.3.1.exe [subcommand] [options]

SUBCOMMANDS:

  abi       - List package ABI, check for ABI breakages.
  build     - Build files, dirs or packages.
  configure - Create build script.
  create    - Create different projects.
  fetch     - Fetch sources.
  generate  - Generate IDE projects.
  install   - Add package to lock.
  integrate - Integrate sw into different tools.
  list      - List packages in database.
  open      - Open package directory.
  override  - Override packages locally.
  remote    - Manage remotes.
  remove    - Remove package.
  run       - Run target (if applicable).
  setup     - Used to do some system setup which may require administrator access.
  test      - Run tests.
  update    - Update lock file.
  upload    - Upload packages.
  uri       - Used to invoke sw application from the website.

  Type "sw.client.sw-0.3.1.exe <subcommand> -help" to get more help on a specific subcommand

OPTIONS:

General options:

  -B                             - Build always
  -D=<string>                    - Input variables
  -activate=<string>             - Activate specific packages
  -build-name=<string>           - Set meaningful build name instead of hash
  -cc-checks-command=<string>    - Automatically execute cc checks command
  -checks-st                     - Perform checks in one thread (for cc)
  -compiler=<string>             - Set compiler
  -config-name=<string>          - Set meaningful config names instead of hashes
  -configuration=<string>        - Set build configuration.
                                   Allowed values:
                                       - debug, d
                                       - release, r
                                       - releasewithdebuginformation, releasewithdebinfo, rwdi
                                       - minimalsizerelease, minsizerel, msr
                                   Default is release.
                                   Specify multiple using a comma: "d,r".
  -curl-verbose                  -
  -d=<path>                      - Working directory
  -debug-configs                 - Build configs in debug mode
  -do-not-mangle-object-names    -
  -do-not-remove-bad-module      -
  -exclude-target=<string>       - Targets to ignore
  -explain-outdated              - Explain outdated commands
  -explain-outdated-full         - Explain outdated commands with more info
  -host-cygwin                   - When on cygwin, allow it as host
  -host-settings-file=<path>     - Read host settings from file
  -ignore-source-files-errors    - Useful for debugging
  -ignore-ssl-checks             -
  -j=<int>                       - Number of jobs
  -k=<int>                       - Skip errors
  -l                             - Use lock file
  -libc=<string>                 - Set build libc
  -libcpp=<string>               - Set build libcpp
  -list-predefined-targets       - List predefined targets
  -list-programs                 - List available programs on the system
  -log-to-file                   -
  -os=<string>                   - Set build target os
  -platform=<string>             - Set build platform.
                                   Examples: x86, x64, arm, arm64
  -print-checks                  - Save extended checks info to file
  -r=<string>                    - Select default remote
  -s                             - Force server resolving
  -save-all-commands             -
  -save-command-format=<string>  - Explicitly set saved command format (bat or sh)
  -save-executed-commands        -
  -save-failed-commands          -
  -sd                            - Force server db check
  -self-upgrade                  - Upgrade client
  -settings=<string>             - Set settings directly
  -settings-file=<path>          - Read settings from file
  -settings-file-config=<string> - Select settings from file
  -settings-json=<string>        - Read settings from json string
  -shared-build                  - Set shared build (default)
  -show-output                   -
  -standalone                    - Build standalone binaries
  -static-build                  - Set static build
  -static-dependencies           - Build static dependencies of inputs
  -storage-dir=<path>            -
  -target=<string>               - Targets to build
  -target-os=<string>            -
  -time-trace                    - Record chrome time trace events
  -toolset=<string>              - Set VS generator toolset
  -trace                         - Trace output
  -verbose                       - Verbose output
  -wait-for-cc-checks            - Do not exit on missing cc checks, wait for user input
  -win-md                        - Set /MD build (default)
  -win-mt                        - Set /MT build
  -write-output-to-file          -

Generic Options:

  -help                          - Display available options (-help-hidden for more)
  -help-list                     - Display list of available options (-help-list-hidden for more)
  -version                       - Display the version of this program
\end{verbatim}





\section{Internals}
\subsection{Components and directory structure}

Components

\begin{enumerate}
\item
Support - small helper library.
\item
Manager - does package managing: package paths, package ids, local databases, remote repository selector, package downloader etc.
\item
Builder - controls executed commands, file timestamps, timestamp databases.
\item
Driver.cpp - main program driver that implements high level build, solutions, targets representation, programs (compilers), source files, languages and generators.
\item
Client - main executable.
\end{enumerate}

Directory structure

\begin{enumerate}
\item
doc - documentation
\item
include - some public headers; not well set up
\item
src - source code
\item
test - tests
\item
utils - all other misc. files: crosscompilation configs etc.
\end{enumerate}
