% !TEX root = sw.tex
% !TeX spellcheck = en_US


\chapter{Client tools (SW and others)}

Main client tool is SW.

A gui frontend is also available. It is Qt-based and widgets-based (no QML).

\section{Quick Start}



\subsection{Package Management}

SW uses packages paths to uniquely identify projects on the web.
Project path is similar to Java packages. Version is appended to it.

`org.username.path.to.project.target-version`

It contains:
\begin{enumerate}
\item
Namespace: `org`
\item
User name: `username`
\item
Path to target: `path.to.project.target`
\end{enumerate}

Allowed namespaces are:
\begin{enumerate}
\item
`org` - organization's public projects with open (free) software license.
\item
`com` - organization's projects with commercial license.
\item
`pub` - user's public projects.
\item
`pvt` - user's private projects.
\end{enumerate}

`version` - is a **semver-like** version.

\subsection{Build process}

SW uses configuration files to describe build process (Build).

Build contains several solutions (Solution).
Each Solution is responsible for exactly one configuration. For example, x86+dll+Release or arm64+static+Debug (plus sign is used only for this document).

Solution consists of targets, projects and directories. Each object might include any other object.

SW uses multipass approach if it needs to build more than one solution.
(Unlike CMake that is telled which options go to what configuration during single pass.)

It is possible to use different frontends to describe build.
Main frontend is called `sw` as well as the program and uses C++ code directly.
Current secondary frontend is `cppan` YAML frontend used in first version of the program.

Build process

\subsection{Frontends}

\subsubsection{C++ frontend (sw)}

HINT: run `sw create project` to force sw to create first config and generate an IDE project for you._

To start your first build configuration, create a file called `sw.cpp`.

\begin{swcpp}
// this is our 'main' build function signature
// function accepts current Solution
void build(Solution &s)
{
    // add non copyable target to Solution of type Executable with name 'mytarget'
    auto &t = s.addTarget<Executable>("mytarget");
    t += "src/main.cpp"; // add our first source file
    // t += "dependency.path"_dep; // add needed dependency
}
\end{swcpp}

SW uses the latest current standard available - C++17.

To control your build (add definitions, dependencies, link libraries) sw uses user-defined literals. For more help on this, see https://github.com/SoftwareNetwork/sw/blob/master/src/sw/driver/suffix.h
Examples:

\begin{swcpp}
t += "MY_DEFINITION"_def;
t += "ws2_32.lib"_slib;
t += "src"_idir;
\end{swcpp}

Now create `src/main.cpp` file with contents you want.

Run `sw build` to perform build or `sw generate` to generate IDE project.



Visibility

To provide control on properies visibility, one might use following target members:

\begin{swcpp}
t.Private += ...; // equals to t += ...;
t.Protected += ...; // available only for this target and current project members
t.Public += ...; // available for this target and all downstream users
t.Inheritance += ...; // available only for all downstream users
\end{swcpp}

Underlying implementation uses three bits to describe inheritance.

\begin{enumerate}
\item This target T
\item This project P (excluding T)
\item All other projects A (excluding P and T)
\end{enumerate}

This gives 7 different inheritance modes (excluding 000) and you could use them all!





More than one configuration

To create additional configs you might add them on command line or write into configuration.
Add following code to your `sw.cpp`:

\begin{swcpp}
// configure signature
// Note that it accepts whole build, not a single solution!
void configure(Build &b)
{
    auto &s1 = b.addSolution(); // add first solution
    s1.Settings.Native.LibrariesType = LibraryType::Static;
    s1.Settings.Native.ConfigurationType = ConfigurationType::ReleaseWithDebugInformation;

    auto &s2 = b.addSolution(); // add first solution
    s2.Settings.Native.LibrariesType = LibraryType::Shared;
    s2.Settings.Native.ConfigurationType = ConfigurationType::Debug;

    return;

    // following code can be used for advanced toolchain configuration
    if (b.isConfigSelected("cygwin2macos"))
        b.loadModule("utils/cc/cygwin2macos.cpp").call<void(Solution&)>("configure", b);
    else if (b.isConfigSelected("win2macos"))
        b.loadModule("utils/cc/win2macos.cpp").call<void(Solution&)>("configure", b);
}
\end{swcpp}

In this function you could provide full custom toolchain to use (compilers, linkers, librarians etc.).
Extended configs can be found here: \url{https://github.com/SoftwareNetwork/sw/tree/master/utils/cc}


\subsubsection{YAML declarative frontend (CPPAN)}

CPPAN is a declarative frontend using YAML syntax.

Example: \url{https://github.com/cppan/cppan/blob/v1/doc/cppan.yml}

Quick Start

We'll describe initial setup and usage based on this demo project \url{https://github.com/cppan/demo_project}.

To start work with cppan install its client into the system. On all systems and work can be done without root privileges.
\begin{enumerate}
\item
Create your project's initial structure, `CMakeLists.txt` file etc.
\item
Create `cppan.yml` file in the project's root directory.
\item
Open it and write dependencies on which your project depends.
\end{enumerate}

`cppan.yml` files use YAML syntax which is very simple. Example of this file can be:

\begin{cppan}
    dependencies:
        pvt.cppan.demo.gsl: master
        pvt.cppan.demo.yaml_cpp: "*"

        pvt.cppan.demo.boost.asio: 1
        pvt.cppan.demo.boost.filesystem: 1.60
        pvt.cppan.demo.boost.log: 1.60.0
\end{cppan}

`dependencies` directive tells the CPPAN which projects are used by your project. `*` means any latest fixed (not a branch) version. Textual name is a branch name (`master`). Versions can be `1.2.8` - exact version, `1.2` means `1.2.*` any version in 1.2 series, `1` means `1.*.*` and version in 1 series. When the new version is available it will be downloaded and replace your current one only in case if you use series versions. If you use strict fixed version, it won't be updated, so no surprises.

Now you should run `cppan` client in project's root directory where `cppan.yml` is located. It will download necessary packages and do initial build system setup.

After this you should include `.cppan` subdirectory in your `CMakeLists.txt` file.

\begin{cmake}
    add_subdirectory(.cppan)
\end{cmake}

You can do this after `project()` directive. In this case all dependencies won't be re-built after any compile flags. If you need your special compile flags to have influence on deps, write `add_subdirectory(.cppan)` after compiler setup.

For your target(s) add `cppan` to `target_link_libraries()`:

\begin{cmake}
    target_link_libraries(your_target cppan)
\end{cmake}

CMake will try to link all deps into this target. To be more specific you can provide only necessary deps tothis target:

\begin{cmake}
    target_link_libraries(your_target org.boost.algorithm-1.60.0) # or
    target_link_libraries(your_target org.boost.algorithm-1.60) # or
    target_link_libraries(your_target org.boost.algorithm-1) # or
    target_link_libraries(your_target org.boost.algorithm) # or
\end{cmake}

All these names are aliases to full name. So, when you have more that 1 version of library (what is really really bad!), you can specify correct version.
For custom build steps you may use executables by their shortest name.

Internally cppan generate a `CMakeLists.txt` file for dependency. It will use all files it found in the dependency dir.

\begin{cmake}
    file(GLOB_RECURSE src "*")
\end{cmake}


Syntax

files

In `files` directive you specify what files to include into the distributable package. It can be a regex expression or a relative file name.

\begin{cppan}
    files:
        - include/.*
        - src/.*
\end{cppan}

or

\begin{cppan}
    files:
        - sqlite3.h
        - sqlite3ext.h
        - sqlite3.c
\end{cppan}

or just

\begin{cppan}
    files: include/.*
\end{cppan}

or 

\begin{cppan}
    files: # from google.protobuf
      - src/.*\.h
      - src/google/protobuf/arena.cc
      - src/google/protobuf/arenastring.cc
      - src/google/protobuf/extension_set.cc 
\end{cppan}

dependencies

`dependencies` contains a list of all dependencies required by your project. Can be `private` and `public`. `public` are exported when you add you project to C++ Archive Network. `private` stays private and can be used by your project's tools and other stuff.

For example when you develop an application, you want to add unit tests, regression tests, benchmarks etc. You can specify test frameworks as dependencies too. For example, `pvt.cppan.demo.google.googletest.gtest` or `pvt.cppan.demo.google.googletest.gmock`.

But when you develop a library and want export it to CPPAN, you won't those libraries in the public dependency list.
You can write:

\begin{cppan}
    dependencies:
      public:
        org.boost.filesystem: 1
      private:
        pvt.cppan.demo.google.googletest.gtest: master
\end{cppan}

By default all deps are public.

\begin{cppan}
    dependencies:
        org.boost.filesystem: 1.60 # public
        pvt.cppan.demo.google.googletest.gtest: master # public now
\end{cppan}

include_directories

Include directories are needed by your project and project users. You must always write `private` or `public` keywords. `private` include dirs available for your lib only. `public` available for your lib and its users.

\begin{cppan}
    include_directories: # boost.math example
      public:
        - include
      private:
        - src/tr1
\end{cppan}

license

Include license file if you have it in the project tree.

    license: LICENSE.txt

root_directory

When adding version from remote file (archive) often there is a root dir inside the archive. You can use this directive to specify a path to be added to all relative files and dirs.

\begin{cppan}
    root_directory: sqlite-amalgamation-3110000 # sqlite3 example
\end{cppan}

exclude_from_build

Sometimes you want to ship a source file, but do not want to include it into build. Maybe it will be conditionally included from config header or whatever.

\begin{cppan}
    exclude_from_build: # from boost.thread
      - src/pthread/once_atomic.cpp
\end{cppan}

options

In `options` you can provide predefined macros for specific configurations.
`any`, `static` and `shared` are available.
Inside them you can use `definitions` to provide compile defines. You must write `public` or `private` near each define. `private` is only for current library. `public` define will see all users and the current lib.

\begin{cppan}
    options: # from boost.log
        any:
          definitions:
            public: BOOST_LOG_WITHOUT_EVENT_LOG
            private: BOOST_LOG_BUILDING_THE_LIB=1
        shared:
          definitions:
            public: BOOST_LOG_DYN_LINK
            private: BOOST_LOG_DLL
\end{cppan}

But try to minimize use of such options.

pre_sources, post_sources, post_target, post_alias

You can provide your custom build system insertions with these directives.

\begin{cppan}
    post_sources: | # custom step from boost.config
      if (WIN32)
        file(WRITE ${CMAKE_CURRENT_SOURCE_DIR}/include/boost/config/auto_link.hpp "")
      endif() 
\end{cppan}

Can be used in options too near with `definitions`.

root_project, projects

Can be used when you're exporting more than one project from one repository. Inside `projects` you should describe all exported projects. Relative name of each should be specified after this. Inside each project, write usual command.

\begin{cppan}
    root_project: pvt.cppan.demo.google.googletest
      projects:
        gtest: # use relative name here - absolute will be pvt.cppan.demo.google.googletest.gtest
          license: googletest/LICENSE
          files:
            - googletest/include/.*\.h
            - googletest/src/.*
        gmock: # use relative name here
          # ...
\end{cppan}

package_dir

Can be used to choose storage of downloaded packages. Could be `system`, `user`, `local`. Default is `user`. `system` requires root right when running cppan, deps will go to `/usr/share/cppan/packages`. `user` stores deps in `\$HOME/.cppan/packages`. `local` will store deps in cppan/ in local dir.

    package_dir: local

You can selectively choose what deps should go to one of those locations.

\begin{cppan}
    dependencies:
        org.boost.filesystem:
            version: "*"
            package_dir: local
\end{cppan}

It is convenient when you want to apply your patches to some dependency.

check_function_exists, check_include_exists, check_type_size, check_symbol_exists, check_library_exists

These use cmake facilities to provide a way of checking headers, symbols, functions etc. Cannot be under `projects` directive. Can be only under root. They will be gathered during `cppan` run, duplicates will be removed, so not duplicate work.

\begin{cppan}
    check_function_exists: # from libressl
      - asprintf
      - inet_pton
      - reallocarray

    check_include_exists:
      - err.h
\end{cppan}






\section{Command Line Reference}




\section{Build System Supported Languages}

SW aims to build a wide specter of programming languages.

Mature support:
\begin{enumerate}
\item ASM
\item C
\item C++
\end{enumerate}

Hello-world support:
\begin{enumerate}
\item C\#
\item D
\item Fortran
\item Go
\item Java
\item Kotlin
\item Rust
\end{enumerate}


Ported Intepreters

- perl (partial)

- python

- tcl

Work started

- java


\section{Integrations}

It is possible to use SW as package manager for your favourite build system.

Here you can find list of completed integrations with examples.

\subsection{CMake}


Full example:\\
\url{https://github.com/SoftwareNetwork/sw/blob/master/test/integrations/CMakeLists.txt}

Steps
\begin{enumerate}
\item
Download SW, unpack and add to PATH.
\item
Run `sw setup` to perform initial integration into system.
\item
Add necessary parts to your CMakeLists.txt (see below).
\end{enumerate}

First, make sure CMake is able to found SW package:
\begin{cmake}
find_package(SW REQUIRED)
\end{cmake}

Then add necessary dependencies (packages) with
\begin{cmake}
sw_add_package()
\end{cmake}

You can omit version. In this case the latest will be used.

Next step is to execute SW to prepare imported targets script with
\begin{cmake}
sw_execute()
\end{cmake}

Complete example:
\begin{cmake}
find_package(SW REQUIRED)
sw_add_package(
    org.sw.demo.sqlite3
    org.sw.demo.glennrp.png
)
sw_execute()
\end{cmake}

Last thing is to add dependency to your package:
\begin{cmake}
add_executable(mytarget ${MY_SOURCES})
target_link_libraries(mytarget 
    org.sw.demo.glennrp.png
)
\end{cmake}

Configure SW deps

Set SW_BUILD_SHARED_LIBS to 1 to build shared libs instead of static (default)
\begin{cmake}
set(SW_BUILD_SHARED_LIBS 1)
\end{cmake}


\subsection{Waf}

Full example\\
\url{https://github.com/SoftwareNetwork/sw/blob/master/test/integrations/wscript}

\begin{enumerate}
\item
Copy-paste these lines to your file \url{https://github.com/SoftwareNetwork/sw/blob/master/test/integrations/wscript#L4-L25}

\item
Replace these lines with your dependencies \url{https://github.com/SoftwareNetwork/sw/blob/master/test/integrations/wscript#L9-L10}

\item
Use deps like on those lines \url{https://github.com/SoftwareNetwork/sw/blob/master/test/integrations/wscript#L35-L36}

\end{enumerate}

To build things call

\begin{command}
waf sw_configure sw_build configure build
\end{command}

sw\underline\ configure step setups SW for waf.

sw\underline\ build step builds necessary deps.





\section{Generators}

SW allows you to generate projects to many different existing systems.

\begin{center}
\begin{tabular}{lc}
Generator & Support\\
\hline
Visual Studio (C++ projects) & Full \\
Visual Studio (NMake projects) & Full \\
%\item Visual Studio (Utility projects) [3/5]
%\item Visual Studio (NMake + Utility projects) [3/5]
%CMake [0/5]
Compilation Database & Full \\
Make & Full \\
NMake & 4/5 \\ %(why?)
Ninja & Full \\
%QMake & 0/5 \\
Batch (win) & Full \\
Shell (*nix) & Full \\
SW build description & Full \\
SW execution plan & Full \\
Raw bootstrap & Full \\
\hline

Missing generators: & \\
Xcode & \\
CMake & \\
qmake & \\

\hline
\end{tabular}
\end{center}


Raw bootstrap is used to bootstrap programs without sw.


\section{Reproducible builds}

SW supports reproducible builds.
Invoke your commands with `--reproducible-build` parameter.
Binaries MUST BE the same on all systems assuming you compare identical build configs.

\begin{command}
sw --reproducible-build ... build
\end{command}


\section{Command Line Reference}

\subsection{sw build}

Build current project.

To get build and dependencies graphs, run:
\begin{command}
sw -print-graph build
\end{command}

\subsection{sw generate}

Generate IDE project.

\subsection{sw upload}

Upload current project to Software Network.
Project will be fetched from your version control system using specified tag/version/branch.

\subsection{sw create}

Create templated project.

\subsection{sw -help}

CLI help.

\begin{verbatim}
OVERVIEW: SW: Software Network Client

  SW is a Universal Package Manager and Build System

USAGE: sw.client.sw-0.3.1.exe [subcommand] [options]

SUBCOMMANDS:

  abi       - List package ABI, check for ABI breakages.
  build     - Build files, dirs or packages.
  configure - Create build script.
  create    - Create different projects.
  fetch     - Fetch sources.
  generate  - Generate IDE projects.
  install   - Add package to lock.
  integrate - Integrate sw into different tools.
  list      - List packages in database.
  open      - Open package directory.
  override  - Override packages locally.
  remote    - Manage remotes.
  remove    - Remove package.
  run       - Run target (if applicable).
  setup     - Used to do some system setup which may require administrator access.
  test      - Run tests.
  update    - Update lock file.
  upload    - Upload packages.
  uri       - Used to invoke sw application from the website.

  Type "sw.client.sw-0.3.1.exe <subcommand> -help" to get more help on a specific subcommand

OPTIONS:

General options:

  -B                             - Build always
  -D=<string>                    - Input variables
  -activate=<string>             - Activate specific packages
  -build-name=<string>           - Set meaningful build name instead of hash
  -cc-checks-command=<string>    - Automatically execute cc checks command
  -checks-st                     - Perform checks in one thread (for cc)
  -compiler=<string>             - Set compiler
  -config-name=<string>          - Set meaningful config names instead of hashes
  -configuration=<string>        - Set build configuration.
                                   Allowed values:
                                       - debug, d
                                       - release, r
                                       - releasewithdebuginformation, releasewithdebinfo, rwdi
                                       - minimalsizerelease, minsizerel, msr
                                   Default is release.
                                   Specify multiple using a comma: "d,r".
  -curl-verbose                  -
  -d=<path>                      - Working directory
  -debug-configs                 - Build configs in debug mode
  -do-not-mangle-object-names    -
  -do-not-remove-bad-module      -
  -exclude-target=<string>       - Targets to ignore
  -explain-outdated              - Explain outdated commands
  -explain-outdated-full         - Explain outdated commands with more info
  -host-cygwin                   - When on cygwin, allow it as host
  -host-settings-file=<path>     - Read host settings from file
  -ignore-source-files-errors    - Useful for debugging
  -ignore-ssl-checks             -
  -j=<int>                       - Number of jobs
  -k=<int>                       - Skip errors
  -l                             - Use lock file
  -libc=<string>                 - Set build libc
  -libcpp=<string>               - Set build libcpp
  -list-predefined-targets       - List predefined targets
  -list-programs                 - List available programs on the system
  -log-to-file                   -
  -os=<string>                   - Set build target os
  -platform=<string>             - Set build platform.
                                   Examples: x86, x64, arm, arm64
  -print-checks                  - Save extended checks info to file
  -r=<string>                    - Select default remote
  -s                             - Force server resolving
  -save-all-commands             -
  -save-command-format=<string>  - Explicitly set saved command format (bat or sh)
  -save-executed-commands        -
  -save-failed-commands          -
  -sd                            - Force server db check
  -self-upgrade                  - Upgrade client
  -settings=<string>             - Set settings directly
  -settings-file=<path>          - Read settings from file
  -settings-file-config=<string> - Select settings from file
  -settings-json=<string>        - Read settings from json string
  -shared-build                  - Set shared build (default)
  -show-output                   -
  -standalone                    - Build standalone binaries
  -static-build                  - Set static build
  -static-dependencies           - Build static dependencies of inputs
  -storage-dir=<path>            -
  -target=<string>               - Targets to build
  -target-os=<string>            -
  -time-trace                    - Record chrome time trace events
  -toolset=<string>              - Set VS generator toolset
  -trace                         - Trace output
  -verbose                       - Verbose output
  -wait-for-cc-checks            - Do not exit on missing cc checks, wait for user input
  -win-md                        - Set /MD build (default)
  -win-mt                        - Set /MT build
  -write-output-to-file          -

Generic Options:

  -help                          - Display available options (-help-hidden for more)
  -help-list                     - Display list of available options (-help-list-hidden for more)
  -version                       - Display the version of this program
\end{verbatim}


\section{Work behind proxy}

To use sw client via proxy, add following to your SW YAML config in
\textasciitilde/.sw/sw.yml:
\begin{verbatim}
proxy:
    host: hostname:port
    user: user:passwd
\end{verbatim}

\section{Bootstrapping}

SW client is distributed in a form of precompiled binary.
This binary is built with SW tool, so we have a circular dependency.

It is important to be able to build it without existing SW client.
Here comes bootstrapping procedure.
To make bootstrap package, use `rawbootstrap` generator:
\begin{command}
sw *needed config* generate -g rawbootstrap
\end{command}






\section{Internals}
\subsection{Components and directory structure}

Components

\begin{enumerate}
\item
Support - small helper library.
\item
Manager - does package managing: package paths, package ids, local databases, remote repository selector, package downloader etc.
\item
Builder - controls executed commands, file timestamps, timestamp databases.
\item
Driver.cpp - main program driver that implements high level build, solutions, targets representation, programs (compilers), source files, languages and generators.
\item
Client - main executable.
\end{enumerate}

Directory structure

\begin{enumerate}
\item
doc - documentation
\item
include - some public headers; not well set up
\item
src - source code
\item
test - tests
\item
utils - all other misc. files: crosscompilation configs etc.
\end{enumerate}
